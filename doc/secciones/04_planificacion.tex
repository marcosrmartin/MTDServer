\chapter{Planificación}
En este capítulo se abordarán aspectos esenciales para el desarrollo del proyecto. Se explicara el \textit{midset} y metodología utilizados para establecer y desarrollar los objetivos. Se presentan las historias de usuario que guiarán el desarrollo e hitos como objetivos concretos y como cumplirlos, por último se estiman los costos del proyecto, incluyendo la amortización del hardware. En resumen, este capítulo proporciona una visión general de la estrategia y recursos necesarios para llevar a cabo el proyecto de manera eficiente.

\section{Desarrollo ágil}
La realización de este proyecto se ha llevado a cabo con \textbf{desarrollo ágil}. Este se basa en priorizar la entrega de software funcional, plazos de entregas reducidos, colaboración con el cliente y adaptabilidad frente a cambios. Fue propuesto en 2001, en el \textit{Manifiesto Agile}.\cite{agile}

O de una forma más sencilla, esta metodología nos dice:
\begin{enumerate}
    \item ¿Qué tengo que hacer ahora?
    \item ¿Es correcta la solución que he planteado al problema que estoy resolviendo?
\end{enumerate}

\section{Historias de usuario}
Serán aquellas peticiones que guiarán el desarrollo, es decir son las peticiones a satisfacer para tener un desarrollo correcto. Se han creado las siguientes:
\begin{itemize}
    \item \href{https://github.com/marcosrmartin/MTD_Server/issues/72}{[HU-1]}:  
    Como Sysadmin quiero añadir un sistema MTD para complementar la seguridad de mi servidor contra ataques zero-days/nuevas vulnerabilidades que todavía no tengan soporte. Es imprescindible que requiera el menor número recursos posibles, ya que es un servidor de bajas prestaciones, y que sea autocontenido, es decir, que no requiera de maquinas externas para su manejo.    
    \item \href{https://github.com/marcosrmartin/MTD_Server/issues/73}{[HU-2]}: Como investigador necesitaré conocer el estado del arte de los MTD, encontrar una solución que se ajuste a \#72 y confirmar con benchmarks que dicha solución cumple con su función.
\end{itemize}
A partir de estas historias se creará el camino a seguir para el desarrollo del proyecto.

\section{Historias de desarrollo}
Seran aquellas peticiones que ayudaran a desenvolver el desarrollo, no son requisitos del cliente:

\begin{itemize}
    \item \href{https://github.com/marcosrmartin/MTD_Server/issues/16}{HD-1}: Como programador, tengo que preparar el entorno para el desarrollo del \href{https://github.com/marcosrmartin/MTD_Server/commit/20df5bb6f5af3de7e557c254ad47089db34845aa}{MTD}. Entre mis necesidades se encuentran realizar test sobre las implementaciones, gestionar sus dependencias y facilitar el proceso de instalación. Todo esto me permitirá trabajar más cómodamente, a la vez que mejorar el flujo de trabajo.
\end{itemize}

\section{Seguimiento del desarrollo - Hitos}
``Los milestones son herramientas para comenzar a trabajar y organizar el trabajo con un objetivo claro y concreto en cada fase.
''\cite{iv}. Es decir, en cada uno se entregará un producto mínimamente viable, el cual tendrá una serie de requisitos para ser considerado válido.

Aplicando el desarrollo ágil a estos \textit{milestones}, se desarrollarán de forma incremental, es decir que el conjunto de hitos sean una serie de escalones. Esto se hace para reducir el plazo de iteraciones y facilitar la toma de dicisiones. Dichos escalones serán planificados partiendo de las peticiones hechas por los usuarios(HUs).

% \section{Temporización}
% Hacer gráfica cuando se tenga el total de horas

\section{Costes}
\begin{table}[H]
	\centering
	\begin{tabular}{| l | l | r |}
        \hline
        \textbf{Concepto} & \textbf{Materiales} & \textbf{Precio} \\
        \hline
        Hardware	& Ordenador y periféricos & Amortización* 308.75 €/año\\
        Personal 	& Ingeniero Junior	& 18000-23000 al año \\
        Software 	& Software libre gratuito & 0 € \\
        % Recursos en la nube 	& GitHub plan gratuito & 0 € \\
        \hline
        % Coste total 			& x horas de desarrollo & x-x € \\
        \hline
	\end{tabular}
	\caption{Costes estimados del proyecto.}
\end{table}

% https://twitter.com/isagasti/status/1593923910617251841
*Amortización aplicando el coeficiente máximo de amortización lineal para el grupo ''equipos para procesos de información''\cite{amortizacion}. Se incluye todo el equipo informático como conjunto operativo. Coste de compra total 1235 €, 1100 € ordenador y 135 € periféricos.


% Ir añadiendo con las implementaciones y toma de dicisiones.
\section{Herramientas}
A continuación se muestra un listado con las herramientas utilizadas en el desarrollo del proyecto: