\chapter{Planificación}

\section{Metodología utilizada - Desarrollo ágil}
La realización de este proyecto se ha llevado a cabo con \textbf{desarrollo ágil}, según Deisy Villalba: ``El desarrollo ágil de software es una forma de integrar las capacidades de un equipo de desarrollo para poner en marcha un plan de trabajo que permita ir entregando pequeños resultados al cliente en poco tiempo, para que este se sienta tranquilo y a gusto con el trabajo realizado. Además, se prima el flujo de trabajo, la colaboración entre los miembros del equipo y siempre enfocarse en lo que se está haciendo y cómo se está haciendo.''\cite{desarrollo-agil}.

O de una forma más sencilla, esta metodología nos dice:
\begin{enumerate}
    \item ¿Qué tengo que hacer ahora?
    \item ¿Es correcta la solución que he planteado al problema que estoy resolviendo?
\end{enumerate}

\section{Historias de usuario}
Serán aquellas peticiones que guiarán el desarrollo, es decir son las peticiones a satisfacer para tener un desarrollo correcto. Se han creado las siguientes:
\begin{itemize}
    \item [HU-1] Adaptar el entorno a las implementaciones
    \item [HU-2] Implementación MASS.
    \item [HU-3] Implementar MTD K8s en el repositorio.
    \item [HU-n] Comparativa entre MASS y MTD K8s.
\end{itemize}
A partir de estas historias se creará el camino a seguir para el desarrollo del proyecto.

\section{Seguimiento del desarrollo - Hitos}
``Los milestones son herramientas para comenzar a trabajar y organizar el trabajo con un objetivo claro y concreto en cada fase.
''\cite{iv}. Es decir, son el camino que nos marcamos antes de empezar a trabajar y estos tendrán una serie de requisitos para poder superarse. Los hitos están basados en las peticiones hechas por los usuarios(HUs). Se han llevado a cabo los siguientes hitos:
\begin{itemize}
    \item [M-0] Preparar el entorno para la documentación.
    \item [M-1] Infraestructura inicial MASS.
    \item [M-0] Implementación MASS.
    \item [M-x] Implementar MTD K8s en repositorio.
    \item [M-x] Comparativa entre MTD K8s y MASS.
\end{itemize}

% \section{Temporización}
% Hacer gráfica cuando se tenga el total de horas

\section{Costes}
\begin{table}[H]
	\centering
	\begin{tabular}{| l | l | r |}
        \hline
        \textbf{Concepto} & \textbf{Materiales} & \textbf{Precio} \\
        \hline
        Hardware	& Ordenador y periféricos & Amortización* 308.75 €/año\\
        Personal 	& Ingeniero Junior	& 18000-23000 al año \\
        Software 	& Software libre gratuito & 0 € \\
        % Recursos en la nube 	& GitHub plan gratuito & 0 € \\
        \hline
        % Coste total 			& x horas de desarrollo & x-x € \\
        \hline
	\end{tabular}
	\caption{Costes estimados del proyecto.}
\end{table}

% https://twitter.com/isagasti/status/1593923910617251841
*Amortización aplicando el coeficiente máximo de amortización lineal para el grupo ''equipos para procesos de información''\cite{amortizacion}. Se incluye todo el equipo informático como conjunto operativo. Coste de compra total 1235 €, 1100 € ordenador y 135 € periféricos.